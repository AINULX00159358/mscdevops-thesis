\subsection{Distinction from PAAS}
Platform as a Service (or PaaS) is a similar cloud offering as FaaS and Serverless. PaaS also abstracts the “servers” and its infrastructure details from developers, making them concentrate only on the business logic. A very common example of such PaaS is Pivotal Cloud Foundry (or PCF). 
The major distinction between PaaS from FaaS is that the former will at least require one running Application Server and its infrastructure. 

\begin{quote}
\textit{
    The key operational difference between FaaS and PaaS is scaling. With most PaaS’s you still need to think about scale, e.g. with Heroku how many Dynos you want to run. With a FaaS application, this is completely transparent. Even if you set up your PaaS application to auto-scale you won’t be doing this to the level of individual requests (unless you have a very specifically shaped traffic profile), and so a FaaS application is much more efficient when it comes to costs.} \\ 
    ( \cite{Roberts_Mike_2018} )
\end{quote}
